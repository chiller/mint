
%----------------------------------------------------------------------------
\section{Továbbfejlesztési lehetőségek}
%----------------------------------------------------------------------------
Derp.

a viz model kibovitese tetszoleges elemekkel

sema refaktor

websockets vs api

\subsection{Shell a böngészőben}

Jelenleg kipróbálni egy shellben a generált kódot csak a rendszer parancssorában lehet NodeJS Repl shellben vagy Chrome illetve Firefox böngészőben a fejlesztői eszközök közt lehet. Mégjobb lenne, ha a webalkalmazás része lenne a shell, ahogyan a templateszerkesztés is része.

Ez valójában nem is bonyolult, a lényege az \lstinline{var result = window.eval(command);} parancs, ahol a \lstinline{command} a shell parancs és \lstinline{result} a kimenete és ezt a kettőt kell beolvasni és kiírni.

granular eventek

extra context

sharejs a szovegdobozra es templatere

\subsection{Undo}

Hasznos lenne visszavonni változásokat, főleg a véletlen törlés miatt. A legegyszerűbb verzióban a törlést úgy lehetne visszavonni, hogy nem is törlődne soha entitás, hanem egy attribútumja jelzi, hogy ő törölt. Kliensoldalon nagyon egyszerű lenne \lstinline{ng-hide="entity.deleted"} segítségével elrejteni a törölteket, hiszen ez a direktíva a modelnek a \lstinline{deleted} attribútuma alapján elrejti az elemet, ha ez az érték \lstinline{false}. 

Ahhoz, hogy a visszavonás is működjön értelmesen a törlési sorrend is fontos. Könnyű megoldás időbélyeget is menteni egy törölt entitáshoz, és visszavonásnál a legfrissebben törölt lenne visszavonva. Ez mind természetesen API hívást és socket üzeneteket is feltételezne.

Szerveroldalon a denormalizálásnál szűrni kell a törölt elemeket.

Ezt a megoldást ki lehet bővíteni általánosabb visszavonásra is, ahol egy entitás módosítást is lehetne visszavonni. Ez működhet hasonló módon, időbélyeg alapon, kiegészítve azzal, hogy az entitás kap egy olyan attribútumot, hogy \lstinline{_history} amiben el lehet tárolni az összes előző állapotát. Ezt esetleg nem muszáj a többi felhasználónak is elküldeni, a többi kliens is fel tudja építeni a módosítás történetet. Azért kezdeném a nevét az attribútumnak alulvonással, hogy egyértelmű legyen, hogy őt nem kell verziózni (ekkor a többi nem verziózandó attribútumot is át kellene nevezni).

\begin{lstlisting}[caption=Entitás történettel együtt]
  {
    "_id" : ObjectId("52486e8e9b7f14a725000001"),
    "position" : {
        "top" : 312.9618225097656,
        "left" : 607.920166015625
    },
    "title" : "Alszik",
    "_history" :
        [{
            "position" : {
                "top" : 312.9618225097656,
                "left" : 607.920166015625
            },
            "title" : "Alszik nagyon"
        },{
            "position" : {
                "top" : 312.9618225097656,
                "left" : 500.920
            },
            "title" : "Alszik nagyon"
        }],
    "document" : "5242aa48ddda9b0000000001"
}
\end{lstlisting}

\subsection{Template továbbfejlesztés}

Code highlight

\subsection{Aldiagramok}

\subsection{Livecoding}

Livecoding alatt az értendő, hogy a fejlesztőkörnyezet lehetővé teszi, hogy miközben kódolunk folyamatosan látjuk az eredményét a kódnak. Ezt úgy lehet elképzelni, hogy -- példáúl -- szükségünk van egy adattranszformációra bemeneti adatok alapján, először megadunk egy teszt adatot, ez kód érthetőség szempontjából is jó, mert gyakorlatilag egy specifikációja annak, hogy milyen adatot várunk. Elkezdjük a függvény törzsét írni és minden sor eredményét automatikusan látjuk egy másik nézeten a teszt adat alapján.

Ez jóval megkönnyítené a gráftranszformációs template megírását, hiszen a gráf a teszt adat és folyamatosan látnánk a kódfordítás eredményét. Azt is be lehetne építeni a logikába, hogy észrevegye, ha a folyamatos fordítás túl erőforrásigényes és csak időnként fordítaná le. Erre tökéletes eszköz a \lstinline{throttle} függvény az Underscore segédkönyvtárból, hiszen vele be lehet állítani, hogy csak meghatározott időközönként fusson le egy parancs.

Az automatikus kiértékelés az a generált kódra is lehet alkalmazni, ahol ugyancsak megadnánk egy teszt adatot. 

Ezt a funkcionalitást kliensoldalon kellene implementálni a kliens-szerver kommunikáció minimalizálása érdekében. A denormalizálást is lehetne kliens oldalon végezni.