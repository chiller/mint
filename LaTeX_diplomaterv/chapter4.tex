%----------------------------------------------------------------------------
\chapter{Az elvégzett munka}
%----------------------------------------------------------------------------

%----------------------------------------------------------------------------
\section{Rendszerterv}
%----------------------------------------------------------------------------
\begin{figure}[!ht]
\centering
\includegraphics[width=100mm,keepaspectratio]{figures/Rendszerterv.png}\\
\end{figure}

\subsection{Kollaboráció}

Mivel kollaboratív alkalmazásról van szó elsőbbségi szempont az ellentmondó, nagyjából egyidejű változtatások konfliktusait feloldani. Felmerült ötletként a Google Docs-ban alkalmazott Operational Transformation megoldás amiben folyó szöveg egyidejű módosítása van megoldva ``előbb-útóbbi'' konzisztencia elérésével. Az Operational Transformation algoritmust nem triviális implementálni, viszont vannak nyílt forráskódú megoldások rá\footnote{http://www.sharejs.com}. Nem ezen a vonalon indultam el, mert nehézkesnek gondoltam annak a rétegnek a fejlesztését ami a DOM-ban levő gráfot szöveggé- és vissza transzformálja akkor is, ha meg van oldva a különböző kliensek által látott szövegreprezentáció kollaboratív változtatása. 

Erre az alternatíva ami végül kiválasztásra került egy meglehetősen pehelysúlyú megoldás: a gráf módosítását jól meghatározott eseményekként defineálom és ezek az események a többi felhasználóhoz Websocket broadcast üzeneteken keresztül jutnának el. Egy példa esemény egy entitás módosítása: az egyik kliens böngészőjében elhúzunk egy elemet, a kliensoldali kód a szervernek egy kérésben szól a módosításról, a szerver perzisztálja az adatot és Websockets broadcast üzeneten keresztül értesíti a többi résztvevőt. Ilyenkor az esemény üzenet a módosult adatot is tartalmazza. 

Ebben a megoldásban igazából semmi konfliktus feloldás nincsen és a továbbiakban be fogom mutatni, hogy ez nem is feltétlenül szükséges. Először is nézzük meg, hogy milyen jellegű konfliktusokról lehet szó; a közösen manipulált elemeket lehet egyidejűleg létrehozni, törölni és módosítani -- feltételezve két szereplőt -- a következő esetek relevánsak:

\begin{enumerate}
\item { \emph{törlés - törlés}}: Ha majdnem egyszerre ketten törlik ugyanazt az entitást, akkor csak kezelni kell kliens és szerver oldalon, hogy a később érkező törlés művelet ne okozzon hibát. 
\item { \emph{módosítás - módosítás}}: Itt két eset lehetséges:
\begin{enumerate}
\item Különböző attribútumok változnak. Ekkor, ha csak a változásokat küldjük el, akkor triviálisan összefésülhetők a különböző attribútumokra vonatkozó egyidejű változások, ha nem, akkor igazából a következő esetről beszélünk,
\item Ugyanazok az attribútumok változnak. Ekkor sajnos arról beszélünk, hogy, példáúl, egyidejűleg az első felhasználó jobbra 
húzta, a másik felhasználó balra húzta az objektumot és a később történt esemény fog végül érvényesülni. 

Az első kérdés ami felmerül itt az, hogy ez egyáltalán próbléma-e? Próbléma, mert ugyan a mozgatás elvesztése apró művelet és könnyen megismételhető, de, ha egy hosszú szöveg beírásáról van szó egy attribútum mezőbe, akkor nem kívánatos elveszíteni azt, ha a másik résztvevő kitörli a régi változatot. Egyszerű modellek szerkesztésénél mint egy állapot diagramnál a hosszú szövegek beírása nem tipikus próbléma. 

Jó esetben körülbelül 100 ms alatt mindenkihez eljutnak az események, ez azt jelentené, hogy nem lehet annyira gyakori ez a próbléma, hogy egy verziókezelő rendszerhez hasonló megoldással lassítsuk a felhasználók munkafolyamatát.

Egy egyszerű felhasználói élmény javítás egy jelzés lesz, ami úgy néz ki, hogy minden felhasználó látja, hogy ki éppen milyen objektumot jelölt ki színekkel jelölve. Ha valaki más ki szeretné törölni az objektumot aminek a hosszú tartalmát éppen írjuk, akkor a rendszer legalább mutatja, hogy azt valaki kijelölte. Ez a megoldás egyik fő előnye, hogy nem bonyolítja egyáltalán a munkafolyamatot példáúl a zároláshoz képest.

\end{enumerate}
\item { \emph{módosítás - törlés }} Ez a szituáció ugyanaz mint az előző. A fent említett megjelölés egy megoldás, de egy visszavonás művelet megnövekedett komplexitás árán megoldaná teljesen a próblémát.
\end{enumerate}

A létrehozás-törlés, létrehozás-módosítás és létrehozás-létrehozás kombinációk nem fordulhatnak elő ugyanazon az entitáson, hiszen nem lehet olyat módosítani egyidőben amit a másik kliens éppen létrehoz.

Ha a két művelet két különböző csúcsot érint, akkor nincs próbléma, mert triviálisan mindkét művelet érvényesül, ha a egy csúcsot és egy hozzá tartozó élet érint, akkor sincs gond, mert a csúcs törlése maga után vonja az él törlését is. 

\subsubsection{Zárolás}

Szigorú zárolás alatt az értendő, hogy valamilyen algoritmus lehetővé teszi, hogy a felhasználó módosítások elől védje a diagram bizonyos részét és itt a módosítások nem kívánt elveszítéséről van szó általános egyidejű szerkesztés közben. Elképzelhető, hogy van értelme ``véglegesíteni'' egy diagramot és akkor megtilthatjuk zárolással a további módosításokat. Elég könnyű találni próblémákat a részleges zárolás ötletében, talán a legkézenfekvőbb a deadlock kialakulása: ketten egyszerre akarnak egy olyan halmazt módosítani, amit nem lehet egyszerre zárolni és egymást kizárják. Tehát nő a komplexitás azzal, hogy a rendszernek ezt meg kell akadályoznia és azt is meg kell akadályoznia, hogy túl sokáig ne maradon érvényes a zár.  

Az előnye a pehelysúlyú kollaborációs megoldásnak az egyszerűség és a remélhetőleg jobb felhasználói élmény ami abból eredhet, hogy visszaigazolás nélkül módosul a felhasználó felületén a diagram a saját beavatkozása után. Egy olyan alkalmazásnál ami egy nem-kollaboratív offline alkalmazással versenyzik létfontosságú a kis reakcióidő a felületen. Az egyszerűség maga után vonja a könnyebb karbantarthatóságot és kisebb hibalehetőséget.   

A hátránya az, hogy nem garantálja, hogy mindenkinek érvényesülnek vagy megmaradnak a módosításai.

\subsubsection{Eltérések detektálása}

Egy funkció ami jelzi, hogy két felhasználó eltérő diagramot lát holott ugyanazt kellene, hogy lássák -- nevezzük eltérés detektálásnak -- két szempontból is hasznos lesz: egyrészt a teljesítményelemzés során fel lehet használni arra, hogy a kollaborációs megoldás hatékonyságát vizsgáljam különböző paraméterek mellett, másrészt felhasználói élmény szempontjából szükséges, hiszen, ha eltérést vesz észre az algoritmus, akkor a diagram újratöltésével orvosolható a próbléma. 
A kérdés az, hogy mi az amit összehasonlítunk ilyenkor? A teljesítményt és a sávszélességet szem előtt tartva egy egyszerű megoldás egy hash értéket számolni a diagramból majd ezt összevetni a többi kliens által kiszámolt értékkel.

Felmerült a kérdés, hogy a diagram hash logika hol legyen: kliensoldal, szerveroldal vagy mindkettő. Kezdjük ott, hogy a kliensoldalon mindenképpen van rá szükség, viszont a szerveroldalon a hash kiszámolásához mindig kell az egész diagram a memóriában, ez sok queryt jelenthet, másrészt a kliensoldalon úgyis mindig be van töltve az egész diagram. Ha kihagyom a szerver oldalt belőle, akkor Websocket alapú megoldás eléggé kézenfekvőnek tűnik a több kliens összeegyeztetése szempontjából.

Másik kérdés, hogy, hogyan oldható meg az, hogy ugyanabban az időben normális körülmények között előfordulhat, hogy ketten nem ugyanazt a dokumentumot látják átmenetileg. Erről bővebben a ~\ref{subsec:hashref} fejezetben lesz szó.



%----------------------------------------------------------------------------
\section{Adatbázis}
%----------------------------------------------------------------------------

A kliens oldalról nézve egy felhasználói diagram fő elemei az entitások és a köztük lévő kapcsolatok (vagy csúcsok és élek a gráfban). 


\begin{lstlisting}
{
    "_id" : ObjectId("52486e8e9b7f14a725000001"),
    "position" : {
        "top" : 312.9618225097656,
        "left" : 607.920166015625
    },
    "title" : "Alszik",
    "document" : "5242aa48ddda9b0000000001"
}
\end{lstlisting}

Az entities kollekcióban tárolva vannak a diagram entitásai. Lehetne egy kollekcióban tárolni mindent ami egy diagramhoz tartozik, de azért van szükség külön kollekcióban tárolni az entitásokat, mert az egyedi azonosítójukra szükség van a kliens oldalon is. ERROLBOVEBBENAKLIENSOLDALIALKALMAZASBAN
A position attribútum konkrétan a kliensoldali CSS pozícionálásra van használva, a document attribútum egy idegen kulcs lenne, ha relációs adatbázisról beszélnénk, de valóban annak a dokumentumnak az azonosítóját tárolja aminek része.

\begin{lstlisting}
{
    "_id": "524b08b1b7f4db8e68000001",
    "connections": [
    {
        "from": "524b08b4b7f4db8e68000002",
        "to": "524b08bdb7f4db8e68000003",
        "label": "valami"
    },
    ...],
    "entities": [],
}a,
\end{lstlisting}


%----------------------------------------------------------------------------
\section{A szerver oldali alkalmazás}
%----------------------------------------------------------------------------

%----------------------------------------------------------------------------
\subsection{API}
%----------------------------------------------------------------------------

Az AngularJS kliens egy API-n keresztül kommunikál a szerverrel a dinamikus tartalom lekérdezése és mentése céljából. Ez az API JSON adatokat szolgáltat, amit akár módosítás nélkül fel lehet használni a kliens oldalon bemásolva egy AngularJS model-be; ugyanígy az API-nak beküldött új JSON objektumok módosítás nélkül a MongoDB adatbázisba íródnak. Az API több csomagoló függvényből áll, minden entitás típusra egy; egy ilyen csomagolófüggvényben a létrehozás, törlés, módosítás és olvasáshoz implementáltam függvényeket és ezeket a konkrét API-kat require segítségével betöltöm a \lstinline{routes} modul \lstinline{index.js} fájlba, ahol példányosítom. Két függősége is van egy ilyen API modulnak: az adatbázis kapcsolat és a socket szerver példány.

\begin{lstlisting}
    var DocumentApi = require('./api/docs.js');
    module.exports = exports = function(app, db) {
        sa = new SocketAdapter(app);
        api = new DocumentApi(db, sa);
        app.get('/api/docs', api.docs);
        app.get('/api/docs/:id', api.doc);
        app.post('/api/docs', api.adddoc);
        app.put('/api/docs/:id', api.updatedoc);
        app.delete('/api/docs/:id', api.deletedoc);
    ...
    }

\end{lstlisting}

\begin{lstlisting}
module.exports = function (db, sa) {
  var docs = db.collection("docs");

  this.deletedoc = function(req, res) {
    var oid = mongo.BSONPure.ObjectID(req.params.id)
    docs.remove({_id: oid}, function(err, removed){
      res.json(removed);
    })  
  ...

\end{lstlisting}

Amikor id-ra hivatkozunk mindig szükséges egy \lstinline{BSONPure.ObjectID} példányba csomagolni az azonosítót, mert csak úgy lehet felhasználni MongoDB lekérdezésben.

%----------------------------------------------------------------------------
\subsection{Kétirányú kommunikáció}
%----------------------------------------------------------------------------

Ha azt nézzük, hogy egyidejűleg x darab diagramot szerkeszt diagramonként y darab különböző felhasználó, akkor felmerül az a próbléma, hogy hogyan követjük nyomon, hogy milyen eseményeket milyen socket-ekre kell továbbítani, más szóval hogyan történjen a broadcast. Szerencsére a Socket.IO könyvtár ezt megoldja úgynevezett szoba funkcionalitással, egy szobához hozzáadott socketek tudnak broadcast üzenetet küldeni a többi szobabeli socketnek, ráadásul a socketek eltávolítása is átlátszóan történik és nem kell memóriakezelési próblémákkal küzdeni.Így a socket kezelő szerver oldali logika meglehetősen vékony lesz, de ez is a célja a Socket.IO könyvtárnak. 

Mivel minden szerkesztendő diagramhoz tartozik egy szoba, kézenfekvő a diagram azonosítóját beállítani a szoba neveként. A modularitást szem előtt tartva, a socket kezeléssel kapcsolatos funkcionalitást egy SocketAdapter nevű objektumba csomagoltam. A SocketAdapter példány példányosítja a Socket.IO szervert és gondoskodik az új kapcsolatok beállításáról, rajta keresztül lehet hívni Socket.IO függvényeket.

Egy új socket-re eseménykezelőket hoz létre a SocketAdapter és a legfontosabb talán a \lstinline{room:change} esemény, ami egy diagram megnyitásakor, vagy másik diagram-ra való áttéréskor hívódik meg a kliens által. Ekkor a régi szobából átkerül az új szobába a socket. 

\begin{figure}[!ht]
\centering
\includegraphics[width=15cm,keepaspectratio]{figures/collaboration-seq.png}
\caption{Entitás módosítás feldolgozása}
\label{fig:angularhtml}
\end{figure}




%----------------------------------------------------------------------------
\section{A kliens oldali alkalmazás}
%----------------------------------------------------------------------------



Az AngularJS alkalmazásomnak a fő építőelemei a szolgáltatások, direktívák és kontrollerek. A kontrollerek a DOM egy részéért felelnek összekapcsolva a modellt és a view-t; a direktívák a HTML kiterjesztését valósítják meg új elemekkel, amik újrafelhasználható UI komponenseket hoznak létre; a szolgáltatások ugyancsak újrafelhasználható függőségek, amiket tetszőleges más komponensbe lehet injektálni.

A külső függőségei: 
\begin{enumerate}
\item az AngularJS könyvtár, 
\item az underscoreJS könyvtár ami funkcionális programozási elemekkel bővíti a nyelvet,
\item egy md5 hash-t számoló függvény,
\item Jquery UI a DOM elemek mozgatása miatt,
\item PlumbJS ami DOM elemeket köt össze nyilakkal.
\end{enumerate}

Az első próbálkozásaimban nyilakat a HTML SVG path segítségével próbáltam megvalósítani úgy, hogy adatkötés segítségével frissült a nyíl pozíciója egy elhúzott DOM elemhez képest. Sok funkcionalitást kellett volna implementálnom, így kerestem alternatívákat. A PlumbJS egy élhúzó modul aminek segítségével DOM elemeket meg lehet jelölni mint él kiindulópont vagy végpont és ezt a kettőt összeköti egy SVG alapú vektoros nyíllal. Gondoskodik az új él behúzásáért, nyilvántartja a létrehozott éleket, DOM elem körül egy kerületet kell defineálni és ahhoz kapcsolja az élek végpontjait. Továbbá tud nyilat húzni és cimkét tenni a vonalra; azt a következtetést vontam le, hogy mindent tud, amire szükségem van.  

\subsubsection{Jade markup}

A Jade egy olyan markup nyelv ami HTML-t helyettesít, az a fő különbség, hogy a fa nem XML alapú, hanem a gyerekek indentációval különülnek el a szülőtől, ez amúgy is általánosságban szokás, viszont így nincs rá szükség, hogy be is zárjunk HTML elemeket. Az eredmény karakterszámon spórol és olvashatóbb markupot eredményez. A Jade template a Node alkalmazásban fordul le HTML tartalommá. A Jade valóban template rendszer is, tartalmaz példáúl elégazásokat, de én csak markupként használtam, a logika az az Angular kontrollerekre van bízva.

Template öröklés úgy működik, hogy az \lstinline{extends} paranccsal megjelöljük, hogy ki melyik template a szülő template. A szülőben \lstinline{block} elemeket hozhatunk létre és az ezekben levő DOM rész felülírható a gyerek template-ben. 

Egy hasznos funkció benne a kikommentezés, ami úgy használandó mint a Javascript egy soros komment, viszont a szülő kikommentezése a gyerekek kikommentezését is maga után vonja, így se két helyen nem kell kommentezési parancsot tenni a markupba (a HTML esetében nyitó és záró komment tag kellett) se kiválasztani nem kell a kommentezendő DOM részt.

\begin{lstlisting}
table
  tr
    td(style='width: '+(100/2)+'%').
      Twitter
    td Facebook
\caption{Példa Jade markup}
\end{lstlisting}

\begin{lstlisting}
<table>
  <tr>
    <td style="width: 50%">Twitter</td>
    <td>Facebook</td>
  </tr>
</table>
\caption{Az ekvivalens HTML}
\end{lstlisting}

%----------------------------------------------------------------------------
\subsection{Direktívák}
%----------------------------------------------------------------------------

\subsubsection{ntDraggable}

Az ntDraggable direktíva a mozgatható diagram dobozokat valósítja és a 

\begin{lstlisting}
<div ng-repeat="entity in shared_document.entities" nt-draggable="nt-draggable" class="drag"></div>
\end{lstlisting}


kód szúrja be őket a DOM-ba. Az adatkötésnek köszönhetően, ha az entities tömb bővül, vagy ritkul, vagy egy eleme megváltozik, akkor ez a DOM-ban is érvényesül. Ez a direktíva egy HTML template-t is meghatároz a mozgatható entitásnak, gyakorlatilag olyan mintha az eredeti DOM-ban lenne ez a HTML, de, ha új helyre szeretnénk másolni a funkcionalitást nem kell lemásolni. A template a következőket tartalmazza:

\pdfcomment[icon=Note]{insert: Kep a direktivarol}

\pdfcomment[icon=Note]{insert: Kep a függőségi gráfról}

\begin{enumerate}
\item Egy gyerek div ami a \lstinline|{{entity.title}}| kötésen keresztül a doboz nevét írja ki,
\item A model megváltozása esetében a pozíciót is tudom frissíteni CSS attribútumra való kötéssel: \lstinline| style='top:{{entity.position.top}}px;left:{{entity.position.left}}px'|
\item Egy entitás kiválasztása esetében egy CSS osztályt kell hozzáadni a szülőhöz, jelezva , hogy ki van választva. Az AngularJS szolgáltat erre egy \lstinline{ng-class} direktívát ami feltételes CSS class beállítást valósít meg. 
\item \lstinline|id={{entity._id}}| Az entitás adabázis azonosítója egy az egyben a DOM id lesz -- egy nagyon kézenfekvő megoldás.
\item A többi felhasználó kijelölését egy kis doboz jelöli, ami a felhasználóhoz rendelt színnel lesz kiszínezve.
\end{enumerate}

Az ntDraggable direktíva egy \lstinline{link} függvényt is meghatároz, ennek a törzse pont az a hely ahova a DOM manipuláció és eseménykezelő logikát akarjuk tenni -- pontosabban itt az egérrel való tologatást állítom be. Itt látszik, hogy ez a DOM manipuláció függetlenítve van a prezentációs rétegtől és az alkalmazás többi részétől, ami tesztelhetőség és újrafelhasználás szempontjából értelmes. 

Ahhoz, hogy a div mozgatása a jsPlumb élek mozgatását vonja maga után nem \lstinline{jQuery.draggable}-t kell használni, hanem \lstinline{jsPlumb.draggable}-t, ami kiterjeszti az előbbit. A DOM irányából a kötés a modelhez csak a \lstinline{stop} draggable eseményre történik, mert nem akarunk minden egérmozgatásról értesíteni a többi felhasználót, az nem csak hogy pazarló lenne, hanem zavaró is, így csak akkor terjed a módosítás , ha új helyre engedjük el a dobozt.

Az élhúzás egyszerűség kedvéért ugyancsak egér kattintás és húzogatással történik, de nem a dobozt húzzuk, hanem a nevét, ekkor a jsPlumb gondoskodik az új ideiglenes lógó él kirajzolásáról. Ehhez jsPlumb \lstinline{makeSource} és \lstinline{makeTarget} parancsokat kell hívni, ezek beállítják a cím dobozokat mint cél és forrás a jsPlumb élekhez.

Megjegyzendő, hogy direktívák használata miatt a kontroller és a DOM template nincs csatolva a jsPlumb modulhoz, tehát, ha lecserélnénk a jsPlumb megoldást egy sajátra, ez remélhetőleg csak a direktíva módosítását vonja maga után.

\subsubsection{saveSelected}

A szerkesztő kontrollere nyilvántartja a kiválasztott entitást a gráfban, egy korábbi verzióban kettőt is lehetett kiválasztani, majd ezeket össze lehetett gombnyomással kapcsolni éllel, de ez helyettesítve lett a fennebb említett felhasználói szempontból egyszerűbb megoldással. 

Kijelöléskor  
\pdfcomment[icon=Note]{insert: Kep az attributes pane-rol}
az entitás attribútumai megjelennek a részlet nézeten és itt lehet módosítani példáúl a címét. Az adatkötés elintézi, hogy az input doboz tartalma megegyezzen mindkét irányba a modellel, de ettől még nem fognak propagálódni a módosítások a szerverre. 

Ez a direktíva azt csinálja meg, hogy amilyen input elemhez csatoljuk, annak a fókusz elvesztése a kijelölt elem szerverre való mentését váltja ki. A \lstinline{saveSelectedCon} direktíva hasonlóan élek szerverre mentését váltja ki.


\subsubsection{ntDelete}

Az ndDelete direktíva a billentyű funkciókért felelős, amiből kettő van: entitás létrehozás (E gomb) és törlés (backspace gomb). A triviális megoldás az, hogy a document elemre egy \lstinline{keydown} eseménykezelőt állítunk be. Ez sajnos nem elég, mert, ha éppen szöveget szerkesztünk az oldalon, akkor arra is reagálni fog. A direktíva eseménykezelője megkülönbözteti, hogy az eseményt fogadó elem input vagy textarea elem-e és ha igen, akkor az alapértelmezett működés következik be, egyébként meg a backspace hatására az éppen kijelölt entitást vagy élt kitörli.

Fontos megjegyezni, hogy az olyan callback, ami nem AngularJS része, példáúl egy jQuery eseménykezelő, nem fogja az adatkötés frissítését kiváltani, tehát ha backspace gombnál kitörlünk egy entitást a model lsitából, ez a DOM-ban nem fog érvényesülni, ilyenkor \lstinline{$scope.$apply();} parancsot kell hívni. 

%----------------------------------------------------------------------------
\subsection{Szolgáltatások}
%----------------------------------------------------------------------------

\subsubsection{SocketService}

A websocket kezelés egy tökéletes példa saját AngularJS szolgáltatásra, ezt akár több kontroller, direktíva vagy szolgáltatásba szeretnénk injektálni és fontos lenne a könnyű hozzáférés és a scope frissítés \lstinline{$scope.$apply();} segítségével. Pontosan ez történik, a SocketService kapcsolódik a szerveroldali Socket.IO szerverhez, majd apply blokkban futtatja a socket üzenet küldés (emit) és fogadás (on) callback metódusokat.

Így kontrollerből átlátszóan használom ezt az adapter és ez a következőképpen néz ki kontrolleren belül:

\begin{lstlisting}
 function EditorCtrl($scope,socket, ...) {
    socket.on('entity:create', function (data) {
      var entities = $scope.shared_document.entities;
      entities.push(data.obj);
    });
 }
\end{lstlisting}

\subsubsection{API szolgáltatások}

AngularJS az alacsonyszintű $http szolgáltatást kínálja egyszerű AJAX hívásokra, de, ha egy API-val dolgozunk, akkor a $resource egy olyan absztraktabb szolgáltatás ami REST-szerű API-khoz való kapcsolódást valósít meg. Ekkor saját szolgáltatást kell defineálni egy endpoint megadásával és a műveletek meghatározásával:






\begin{lstlisting}
  app.factory('DocumentService', ['$resource', function($resource){
  return $resource('/api/docs/:id', {id:'@_id'}, 
    { update: {method:'PUT' } , 
      query: {method:'GET', isArray: true}});
    }]);
\end{lstlisting}




Látható, hogy a saját DocumentService API-nek a \$resource a függősége, az \lstinline{/api/docs/:id} URL-hez kapcsolódik és kibővítettem két metódussal ami az update és a query metódus. 





Új entitás létrehozásakor kevés kódot kell futtatni:
\begin{lstlisting}
 $scope.addEntity = function() {
    EntityService.save({ 
      position: {"left":300, top:300 },
      title:"*", 
      document: $scope.shared_document._id}, 
      function(res){}
    )
  }
\end{lstlisting}

Az a kényelmes mellékhatása van ennek a kódnak, hogy új entitás létrehozásakor a DOM-ba kerül.\pdfcomment[icon=Note]{insert: Egybol belekerul a domba?}

\subsubsection{JSPlumbService}


A laza csatolás szem előtt tartása miatt a jsPlumb hívások is egy szolgáltatásban vannak és nincsenek a kontroller vagy view-hoz szorosan csatolva. 

Ez a szolgáltatás beállítja az alapértelmezett jsPlumb paramétereket:

\begin{enumerate}
\item az élek görbülésének mértéke és stílusa,
\item az élek színe, vastagsága,
\item a végpontok stílusa,
\item egy overlay beállítása, ami nyilat valósít meg,
\item  rögzítési pont -- ez esetben folytonos -- ami azt jelenti, hogy nem diszkrét számú pontra illeszkedhet, hanem a doboz kerülete mentén egy közelebbi ponthoz fog kapcsolódni a másik doboz helyzetéhez képest.
\end{enumerate}

Két metódust szolgáltat : új élek létrehozása és egy új diagram betöltésekor  vagy váltáskor az élek felrajzolása egy éllista alapján.
Nincsen adatkötés az élek és az AngularJS modellek között, ezért itt kell ezt megvalósítani.

\subsection{Függőségek vizualizálása}

\begin{figure}[!ht]
\centering
\includegraphics[width=15cm,keepaspectratio]{figures/dependencies.png}
\caption{Függőségek vizualizálása AngularJS chrome toolbar-ban}
\label{fig:angulardependencies}
\end{figure}


\cite{https://github.com/angular/angularjs-batarang/blob/master/README.md}
A Batarang nevű kiegészítés a Google Chrome böngészőhöz egy AngularJS debuggoló eszköz, a ~\ref{fig:angulardependencies} ábrán látszik a függőségek vizualizálása, ami az egyik eszköz a Batarang-ban. A kör mentén fel vannak sorakoztatva a beépített -- általában \$-vel kezdődő nevűek -- és a fejlesztő által implementált komponensek: szolgáltatások, direktívák, filterek. A függőségi gráfban egy komponenst kiválasztva zöld színű élek mutatnak az őt felhasználó komponensekre, a pirosak meg azokra mutat amik őt használják.




\subsection{A szerkesztő kontroller}
\label{subsec:hashref}
\begin{enumerate}
\item A kontroller betöltésekor lekéri a DocumentService-től az összes diagramot és majd az elsőt betölti,
\item defineál egy inicializáló metódust ami a diagram betöltését valósítja meg. Ekkor bizonyos változók újra az eredeti értéküket kapják meg, majd frissülnek az API-ból a diagram adatai, az entitásokat lekéri és inicializálja az éleket. A végén socket üzenetet küld arról, hogy az új diagramra váltott és a szerveroldlon egy másik socket szobába kerül;
\item új diagram létreho
\item 
\item 
\end{enumerate}


documenthash

%----------------------------------------------------------------------------
\subsection{Unit teszt esetek}
%----------------------------------------------------------------------------



%----------------------------------------------------------------------------
\section{Pehelysúlyú zárolás}
%----------------------------------------------------------------------------

Ahelyett, hogy megakadályozzam, hogy a többi résztvevő felhasználó, ne változtathassák bizonyos részeit a diagramnak, csak egy jelzési mechanizmust implementáltam, amivel láthatóvá válik, ha valaki kijelölt egy entitást. Ehhez először szükség van a szoba résztvevők nyilvántartására, hiszen a kollaborációhoz nem volt kötelező tudni, hogy ki nézi még a diagramot.

Szerencsére szerveroldalon a Socket.IO könnyen elérhetővé teszi ezt az információt: \lstinline{io.sockets.clients(szoba);} egy socket objektumokból álló tömböt ad vissza. A \lstinline{connection} és a \lstinline{disconnect} események hatására értesül a szerveroldal arról, hogy valaki kapcsolódott, vagy elhagyta a szobát.

\begin{figure}[!ht]
\centering
\includegraphics[width=15cm,keepaspectratio]{figures/join-seq.png}
\caption{Felhasználó belép a szobába}
\label{fig:joinseq}
\end{figure}

Az ábrán látszik, hogy szoba váltáskor a szerveroldal minden résztvevőjét a szobának (az új résztvevőt beleértve) értesít arról, hogy belépett, és az új résztvevőnek a létező socketek listáját is elküldi. Kilépéskor vagy diagram váltáskor is értesíti a klienseket. Így a kliensek nyilvántartást tudnak tartani a kapcsolódott többi kliensről. A \lstinline{client joined} és a \lstinline{client left} üzenetek esetében -- mivel az alkalmazás nem használ felhasználói fiókokat -- egy socket példány azonosítót is küld a kliensnek.

szinek mint userek
kep a jelolesrol  

162

Színek hogyan?


\section{Vizuális programozási nyelvek és kódgenerálás}

\begin{figure}[!ht]
\centering
\includegraphics[width=15cm,keepaspectratio]{figures/compile-seq.png}
\caption{Kódgenerálás}
\label{fig:compileseq}
\end{figure}

A kollaboratív gráfszerkesztésen túl az alkalmazás egyszerű vizuális programozási nyelvek létrehozására alkalmas.
A vizuális programozás menete a következő egy minta nyelv példányán amiben állapotgépeket lehet programozni:
\begin{enumerate}
\item A kliensoldalon a felhasználó elkészíti a gráftranszformációs template kódot,
\item A felhasználó elkészíti egy vizuális implementációját a nyelvnek, vagyis egy konkrét állapot gépet vizuálisan,
\item A fordítás gomb megnyomása után a szerveralkalmazás denormalizálja a gráfot, azaz létrehoz egy olyan struktúrát, ami a diagramot reprezentálja redundánsan. Ez a struktúra a bemenő adata a gráftranszformáló template-nek,
\item A szerveralkalmazás a template alapján egy konkrét állapotátmenetekkel rendelkező állapotgép Javascript kódját hozza létre,
\item Ezt a kódot elmenti az adatbázisba és választ küld a kliensalkalmazásnak, ekkor kétféleképpen lehet felhasználni a kódot:
\begin{enumerate}
    \item A felhasználó böngészőjének Javascript parancssorában (példáúl Chrome Developer Console) értelmezi a kódot \lstinline{eval} segítségével és példányosítja az állapotgépet. 
    \item Node.JS parancssorból a mintrepl segédalkalmazással betölti a kódot, majd példányosítja az állapotgépet.
\end{enumerate}
\end{enumerate}


\subsubsection{Denormalizálás}

Az eredeti ``séma'' csúcsok listáját és élek listáját jelenti, viszont ez az adatformátum nem hasznos relációk kezelésére. Ezért reláció helyett redundancia bevezetésével oldom meg: egy diagram ilyen reprezentációjában egy él többször is jelenik meg. Ez leegyszerűsítve így néz ki:

\begin{lstlisting}
    {   edges: [
            {from: a, to: b},
            {from: a, to: c}],
        points: [
            {name: a, outgoing: [{from: a, to: b},{from: a, to: c}], incoming: []},
            {name: b, outgoing: [], incoming: [{from: a, to: b}]},
            {name: c, outgoing: [], incoming: [{from: a, to: c}]},
        ]
    }
\end{lstlisting}

Így ezen a struktúrán kevesebb kóddal írható egy olyan transzformáció ami valamilyen logikát valósít meg abból kiindulva, hogy milyen állapotból milyen élek mennek ki és be. Ez struktúra ami ilyenkor létrejön csak átmenetileg van felhasználva amíg kódot generál a szerveroldal, utána nem lesz használva semmire, így nem kella redundancia által okozott szokásos próblémákkal küzdeni. Ebben a formában tárolni mindig a gráfot nem hatékony, mert minden diagram manipulációs művelet több mint egy adatbázis lekérdezést jelentene és ha el szeretném kerülni a redundancia anomáliákat 
akkor megnövekedne fölöslegesen a komplexitás (tranzakciókra lenne szükség, vagy zárolásra és ezeket nem biztosítja a MongoDB). Ez a denormalizálás viszont csak két for ciklusból áll és ritkán fut le. 

\subsubsection{Gráftranszformáció template}

A gráftranszformáció template egy Underscore template. Underscore.JS egy Javascript könyvtár ami funkcionális programozás eszközökkel bővíti e nyelvet, anélkül, hogy kiterjesztené a beépített típusokat. \cite{underscorejs.org}
Ezen belül a template rendszer lehetővé teszi egy template fájlt lefordítását adott kontextus alapján. A template-ben \lstinline{<\%= \%>} segítségével változó behelyettesítés történik, \lstinline{<\% \%>} segítségével Javascript kódot lehet értelmeztetni. 

\begin{lstlisting}
var compiled = _.template("hello: <%= name %>");
compiled({name: 'moe'});
=> "hello: moe"
\end{lstlisting}

Az alkalmazásban így néz ki felhasználói szemszögből:

\begin{figure}[!ht]
\centering
\includegraphics[width=5cm,keepaspectratio]{figures/simple-graph.png}
\caption{Egy példa diagram}
\label{fig:compileseq}
\end{figure}

\begin{lstlisting}[caption=A template]
Elements: 
<% 
    doc.entities.forEach(function(el){
%>
    <%= el.title %>
<%
    })
%>
\end{lstlisting}

És a kimenet: 
\begin{lstlisting}[language=HTML]
Elements: 

source
target
\end{lstlisting}

Természetesen nem csak Javascript kódot lehet generáltatni az alkalmazásban, hiszen a template behelyettesítés eredménye egyszerű szöveg. 






%---------------------------------------------------
\subsubsection{A generált kód felhasználása}
%---------------------------------------------------

A generált kód elmentődik az adatbázisba a diagram egyik attribútumába, és ehhez az adathoz egy API végponton keresztül lehet hozzáférni és felhasználni meg akár a triviális HTML script elem segítségével:

\begin{lstlisting}[language=HTML]
 <script type="text/javascript" src="http://localhost:3000/api/compile/modules/5242aa48ddda9b0000000001/"></script>
\end{lstlisting}

Parancssorból is fel lehet használni \lstinline{node mintrepl <id>} segítségével, a mintrepl egy javascript szkript, ami a fent említett API-n keresztül letölti és értelmezi a fájlt, majd futtatja a letöltött kódot és egy REPL\footnote{Read-Eval-Print-Loop} interaktív node shell-t futtat. 

\begin{lstlisting}
 $ node mintrepl 5242aa48ddda9b0000000001
 mint 5242aa48ddda9b0000000001> tanulo = new sm()
 mint 5242aa48ddda9b0000000001> tanulo.consumeEvent("kávé")
 mint 5242aa48ddda9b0000000001> tanulo.currentState
 ébren
 \end{lstlisting}

 Hasonló módon egy másik node szerver szerveroldali kódjába is be lehet illeszteni. 
 








%----------------------------------------------------------------------------
\section{Az alkalmazás bemutatása}
%----------------------------------------------------------------------------


%----------------------------------------------------------------------------
\section{Továbbfejlesztési lehetőségek}
%----------------------------------------------------------------------------
Derp.


