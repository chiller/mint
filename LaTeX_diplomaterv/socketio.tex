%----------------------------------------------------------------------------
\section{Socket.IO}
%----------------------------------------------------------------------------
A Socket.IO egy JavaScript könyvtár ami valósidejű alkalmazásokat tesz lehetővé. Ha a hagyományos kliens-szerver weboldal modellre gondolunk, akkor azt a hiányosságot oldja meg, hogy miután letöltődik a honlap a böngészőbe nem egyértelmű, hogy hogyan tudnánk a kliens oldalon reagálni szerver oldali eseményekre. A WebSocket protokoll egy TCP kapcsolaton full-duplex kommunikációt biztosít.

Valósidejű alkalmazásra egy naív próbálkozás az egyszerű AJAX polling, ami így nézhet ki:
\medskip
\begin{lstlisting}[caption=Egyszerű polling]
setInterval(function(){
    $.ajax({ url: "server", success: function(data){
        //
    }, dataType: "json"});
}, 30000);
\end{lstlisting}

Ez több próblémát is felvet, többek között azt, hogy, ha a polling intervallum túl kicsi, akkor túl sok lekérés megy a szerverhez, ha az intervallum túl nagy, akkor nem kapunk elég gyakran értesítést. Minden egyes lekérdezés egy-egy kapcsolatfelépítést jelentene, ami pazarló lenne.

Erre egy alternatíva a Long Polling, ekkor a válasz nem érkezik meg a szervertől addig, amíg nincsen olyan esemény amiről a klienst értesítenénk:

\begin{lstlisting}[caption=Long Polling]
(function poll(){
    $.ajax({ url: "server", success: function(data){
        // 
    }, dataType: "json", complete: poll, timeout: 30000 });
})();
\end{lstlisting}

%Itt is láthatunk néhány hiányosságot: újra kell kapcsolódni periodikusan, mert a kérés lejárhat (timeout), a kommunikációs protokoll még mindig HTTP, ami 

Megoldások mint a \url{http://pusher.com} megkönnyítik a Websockets implementációt egy 3rd party szolgáltatás keretein belül.