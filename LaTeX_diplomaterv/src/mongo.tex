%----------------------------------------------------------------------------
\section{MongoDB}
%----------------------------------------------------------------------------

A MongoDB egy nyílt forráskódú dokumentum-orientált adatbázis rendszer. A MongoDB nem egy relációs adatbázis, nem lehet SQL nyelv segítségével lekérdezéseket futtatni, nem támogat JOIN műveletet. A MongoDB BSON\footnote{Binary JSON} formájában tárol kulcs-érték párokat, az RDBMS\footnote{Relational Database Management System} alapfogalmai: adatbázis, tábla, sor, oszlop; a MongoDB megfelelői: adatbázis, kollekció, dokumentum, mező. Egy mező értéke lehet egszerű típus, lista vagy beágyazott dokumentum. A dokumentumok nem támogatnak sémát, tehát azonos fajta információnak lehet eltérő sémájú tartalma. 


\subsection{Indexelés}

MongoDB egy relációs adatbázishoz hasonlóan támogat indexelést, erre egy példa az~\ref{sec:dbprof} fejezetben lesz bemutatva. Érdekes fogalom a ``compound'' vagyis összetett index, ennek segítségével több mezőre is lehet egyszerre indexet építtetni és egy olyan lekérdezés ami egy termék gyártója és típusa alapján szűkít hatékony tud lenni.

Egy másik érdekes index fajta a ``multikey'' index, ami lehetővé teszi, hogy egy tömb értékű mező elemeit is indexeljük. Ha egy blogbejegyzés rekordban a bejegyzést kedvelő felhasználók tömbje van, akkor indexek számára tudunk hatékony lekérdezéseket futtatni, példáúl szűrni az olyan bejegyzéseket, amiket Laci kedvelt.

\subsection{Replikáció}

A MongoDB egyik előnye, hogy nagyon könnyű benne adatbázisreplikációt beállítani. Egy replika halmaz több adatbázist tartalmaz aminek az egyik tagja a fő adatbázis ami elfogad írás műveleteket és a többi tag megkap minden változást és csak olvasás műveletet lehet rajtuk végrehajtani. A fő csomópont kiesése esetében a többi csomópont megszavaz egymás közül egy csomópontot aki átvegye a fő csomópont szerepét. Amikor a régi csomópont feléled, akkor szinkronizálja a változásokat és újra átveszi régi szerepét.

Az a kényelmes, hogy a kapcsolódó adatbázismeghajtók automatikusan kiválasztják a legközelebbi csomópontot az olvasáshoz. A csomópont szavazások és új szerepek kiosztásával nem kell foglalkoznia a fejlesztő aki kapcsolódik egy replika halmazhoz. 

\subsection{Sharding}

Egy nagyon sok adattal dolgozó rendszer túlterhelhet egy MongoDB adatbázist és ezt lehet egy darabig függőleges skálázással orvosolni, vagyis több fizikai erőforrást biztosítunk az adatbázisnak, de ennek is ugyanaz a korlátja, csak magasabban. 

Ehelyett a MongoDB vízszintes skálázhatóságot tesz lehetővé ``sharding'' segítségével. Ekkor az adatok partícionálása történik, a különböző partíciók különböző adatbázisokból lesznek elérhetőek. Az adatok partícionálása egy rögzített mező mentén történik, ezt később nem lehet megváltoztatni. Példáúl egy dokumentumszerkesztő rendszerben érdemes partícionálni a dokumentum tulajdonosa mentén, ekkor egy adatbázisra nézve csökken az írási és olvasási műveletek átlagszáma, ha egy felhasználó munkafolyamatot indít akkor csak az egyik adatbázis partícióban fog dolgozni.

A partícionáló rendszer próbál egyensúlyozottan partícionálni és, ha az egyik partíció megnő, akkor átlátszó módon átmozgat rekordokat egy másik aprtícióba.

A sharding és a replikáció egyszerre használható és kombinálható, példáúl három partíciónk lehet és mind a háromban három-három replika halmazunk lehet, így összesen kilenc csomópontunk lesz.



\subsection{ACID}

Fontos megjegyezni, hogy a RDBMS rendszerekkel szemben a MongoDB nem tud ACID trazakciókat biztosítani\cite{acidref}. 
\begin{enumerate}
\item{Atomicitás} Csak dokumentum szintű atomikus műveletek biztosítottak, ha több kollekció vagy dokumentumon átívelő atomikus tranzakciókat szeretnénk, akkor vagy használjunk RDBMS-t, vagy saját magunk kell egy zárolási mechanizmust fejleszteni.
\item{Konzisztencia} Replikáció esetében a fő adatbázis csomópontot érintik az írás műveletek, az olvasás műveletek meg -- opcionálisan --  a ``közelebb'' tartózkodó csomóponthoz fordulnak. Ebben az esetben ``eventual consistency''-ről beszélünk és nem biztosított, hogy friss adatot olvasunk.  
\item{Izoláció} MongoDB esetében nem lehet beszélni izolációról, mert csak dokumentumszintű tranzakciók vannak és minden dokumentum művelet hatása azonnal elérhető a többi folyamat számára. 
\item{Tartósság} Teljesítmény árán lehet a tartósságot növelni, példáúl úgy, hogy egy írás művelet a többi csomópontra is íródhat mielőtt visszatérne.
\end{enumerate}

%A CAP\footnote{(C)onsistency - Konzisztencia, (A)vailability - Elérhetőség, (P)artition tolerance - Partícionálás tolerancia} tétel kijelenti, hogy egy elosztott rendszer nem tudja egyszerre teljesíteni a ...

A 32 bites rendszereken a mongoDB szerver csak 2GB tárral gazdálkodhat, ugyanakkor a 64 bites esetben csak a hardver mérete korlátozza.





