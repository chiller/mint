%----------------------------------------------------------------------------
\section{AngularJS}
%----------------------------------------------------------------------------

Az AngularJS egy nyílt forrású JavaScript keretrendszer ami ``single-page''\footnote{Egy olyan webalkalmazás ami egy weboldalon kifér dinamikusan betöltött tartalommal a jobb felhasználói élmény céljából.} webalkalmazások fejlesztését támogatja. A Google karbantartása alatt van és az első verziója 2009-ben jelent meg. 

\subsection{Kliensoldali HTML generálás}

A hagyományos weboldal esetében a böngészőben megjelenített HTML a szerver oldali alkalmazás által van létrehozva.  A single-page alkalmazás ebben az esetben úgy jön létre, hogy AJAX felhasználásával betöltés után töltődnek be újabb részei az oldalnak\cite{angularbook}. AngularJS esetében ez a művelet a kliens oldalra kerül így a szerver inkább egy adatforrás és statikus tartalom szolgáltatóvá válik. Így lazább csatolást érünk el a szerver és kliens oldal között esetleg párhuzamosítva a fejlesztést és növelve az újrafelhasználhatóságot. 


\subsection{Direktívák}

Az AngularJS fordító segítségével ki lehet terjeszteni a HTML szintaxist új attribútum és elem típusokkal, ezeket hívják direktívának. 

\lstset{language=HTML}
\begin{lstlisting}[frame=single]  
<ul>
  <li ng-repeat="action in user.actions">
    {{action.description}}
  </li>
</ul>
\end{lstlisting}

\subsection{Model View Controller}

\subsection{Adatkötés}

jQuery és hasonló megoldások segítségével újratöltés nélkül frissíteni tudjuk a DOM-ot felhasználói események hatására vagy adatok frissítése esetében, de még az egyszerű esetekben sem biztos hogy triviális a DOM és adatok összehangolása. Az AngularJS ezt két irányú adatkötéssel próbálja megoldani ráadásul deklaratívan. Egy egyszerű példa adatkötésre:

\lstset{language=HTML}
\begin{lstlisting}[frame=single]  
<div ng-controller="HelloController">
  <input ng-model="greeting.text"/>
  <p>{{greeting.text}}, World!</p>
</div>
\end{lstlisting}
Ekkor az input mező módosítása automatikusan a model-t is frissíti és ez a cimkét is frissíti alatta. Ehhez az alap funkcionalitáshoz nem is szükséges más kódot írni.


\subsection{Függőséginjektálás}


