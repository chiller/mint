%----------------------------------------------------------------------------
\section{Socket.IO}
%----------------------------------------------------------------------------
A Socket.IO egy JavaScript könyvtár ami valósidejű alkalmazások kétirányú kommunikációját támogatja. Ha a hagyományos kliens-szerver weboldal modellre gondolunk, akkor azt a hiányosságot oldja meg, hogy miután letöltődik a honlap a böngészőben, nem egyértelmű, hogy hogyan tudnánk a kliens oldalon reagálni szerver oldali eseményekre. A WebSocket protokoll egy TCP kapcsolaton full-duplex kommunikációt biztosít.

Valósidejű alkalmazásra egy naív próbálkozás az egyszerű AJAX polling, ami így nézhet ki:
\medskip
\begin{lstlisting}[caption=Egyszerű polling]
setInterval(function(){
    $.ajax({ url: "server", success: function(data){
        //
    }, dataType: "json"});
}, 30000);
\end{lstlisting}

Ez több próblémát is felvet, többek között azt, hogy, ha a polling intervallum túl kicsi, akkor túl sok lekérés megy a szerverhez, ha az intervallum túl nagy, akkor nem kapunk elég gyakran értesítést. Minden egyes lekérdezés egy-egy kapcsolatfelépítést jelentene, ami pazarló lehet.

Erre egy alternatíva a Long Polling, ekkor a válasz nem érkezik meg a szervertől addig, amíg nincsen olyan esemény amiről a klienst értesítenénk:

\begin{lstlisting}[caption=Long Polling]
(function poll(){
    $.ajax({ url: "server", success: function(data){
        // 
    }, dataType: "json", complete: poll, timeout: 30000 });
})();
\end{lstlisting}

Itt is láthatunk néhány hiányosságot: újra kell kapcsolódni periodikusan, mert a kérés lejárhat (timeout) és a kommunikációs protokoll még mindig HTTP, ami overhead-et jelent.


A pollingnál egyszerűbb megoldás egy egyetlen TCP kapcsolatot kialakítani a kétirányú kommunikációra, ezt kínálja a Websocket protokoll
\cite{rfc6455}.

A Websocket protokoll arra lett tervezve, hogy a létező kompromisszumos kétirányú megoldásokat helyettesítse. A protokollnal két része van: a handshake és az adatátvitel. Miután végbement a handshake, a két fél \emph{üzenetekben} kommunikál egymással amik \emph{keretekre} lehetnek tagolva.

Ez a keretezés a protokoll saját keretezése és az alsó rétegek akár tovább bonthatják.

A Websocket protokoll egy címzési mechanizmust is megvalósít, amivel több szolgáltatást lehet igénybevenni egy kapcsolaton keresztül.

A kommunikáció a TCP 80 porton történik, ami kedvező akkor amikor tűzfal szűri a többi portot. 




A Socket.IO átlátszó módon kiválasztja a legjobb protokollt az aktuális böngésző által támogatottak közül ebben a sorrendben:
\begin{enumerate}
\item{WebSocket}
\item{Adobe Flash Socket}
\item{AJAX long polling}
\item{AJAX multipart streaming}
\item{Forever Iframe}
\item{JSONP Polling}
\end{enumerate}



Megoldások mint a \url{http://pusher.com} megkönnyítik a Websockets implementációt egy 3rd party szolgáltatás keretein belül, de nem ezt használtam többek között azért mert fizetős szolgáltatás és nem akartam egy külső rendszertől függő alkalmazást fejleszteni.


