%----------------------------------------------------------------------------
\chapter*{Bevezető}\addcontentsline{toc}{chapter}{Bevezető}
%----------------------------------------------------------------------------

Néhány éve foglalkozom Python alapú webfejlesztéssel és munkám során mindig is esztétikusnak tartottam egy könnyen használható de erős keretrendszert. A tapasztalataim elején mindig is óvakodtam a kliensoldali webfejlesztéstől, mert bárhogy csináltam, a jQuery alapú kliensoldali logika a végén sohasem lett olyan szép, mint amire büszke lehetnék. Mire a Javascript tudásom utolérte a Python tudásomat, elburjánoztak a kliensoldali Javascript keretrendszerek amik lehetővé teszik, hogy kliensoldalon is moduláris, tesztelhető és újrafelhasználható megoldásokat alkalmazzak.

A webes modell szerkesztőkre akkor figyeltem fel, amikor a tanulmányaim során a sokadik diagramot kellett megrajzolnom és az akkori Linux rendszeremre nem találtam kielégítő minőségű diagram szerkesztőt. Találtam helyette a Lucidcharts webes eszközt, ami minden akkori próblémát megoldott számomra, ami a diagramszerkesztést illeti.

A vizuális programozásnak az az aspektusa ragadott meg, hogy vannak próblémák amikre egy gráf jellegű megoldás rugalmasabb és átláthatóbb mint a hagyományos kód alapú megoldás és a munkám során próbáltam arra figyelni, hogy olyan vizuális eszközt hozzak létre ami könnyebben old meg próblémákat.

Az MSc szakdolgozatom keretein belül a következőkre terjedt ki a munkám:
\begin{enumerate}
\item Vizuális programozási nyelvek áttekintése és egy vizuális programozási nyelv elkészítése,
\item a kollaboratív diagramszerkesztés megvalósítása, amivel egyidőben több felhasználó tudja ugyanazt a dokumentumot szerkeszteni,
\item a szerveroldali alkalmazás elkészítése, ami perzisztálja az adatokat és támogatja a valósidejű kollaboratív szerkesztést,
\item a HTML5 alapú gráfszerkesztő alkalmazás elkészítése,
\item a megoldáshoz használt technológiák bemutatása, 
\item a megoldás értékelése teljesítmény és skálázhatóság szempontjából.
\end{enumerate}

\clearpage
A fejezetek felépítése a következő:

\begin{enumerate}
\item Az első fejezet a webalkalmazásoknak egy rövid áttekintése, 
\item a második fejezet létező webes diagramszerkesztő, webes programfejlesztő és vizuális programozási nyelveket vizsgál,
\item a harmadik fejezet a kiválasztott technológiákat mutatja be,
\item a negyedik fejezet az elkészült alkalmazás felépítését részletezi mind a szerveroldal és kliensoldali funkciókra kitérve,
\item az ötödik fejezet az alkalmazás teljesítményelemzését mutatja be,
\item az első függelék az alkalmazás telepítését mutatja be Linux rendszeren,
\item a második függelék az alkalmazás felhasználói felületét mutatja be,
\item a harmadik függelék az állapotgép nyelv kódját részletezi.
\end{enumerate}