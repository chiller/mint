%----------------------------------------------------------------------------
\appendix
%----------------------------------------------------------------------------
\chapter*{Függelék}\addcontentsline{toc}{chapter}{Függelék}
\setcounter{chapter}{6}  % a fofejezet-szamlalo az angol ABC 6. betuje (F) lesz
\setcounter{equation}{0} % a fofejezet-szamlalo az angol ABC 6. betuje (F) lesz
\numberwithin{equation}{section}
\numberwithin{figure}{section}
\numberwithin{lstlisting}{section}
%\numberwithin{tabular}{section}

%----------------------------------------------------------------------------
\section{Az alkalmazás telepítése}
%----------------------------------------------------------------------------

Az alkalmazás telepítése példáúl egy Ubuntu Linux rendszeren:

\begin{enumerate}
\item MongoDB adatbázis telepítése a \url{http://docs.mongodb.org/manual/tutorial/install-mongodb-on-ubuntu/} útmutató alapján:
\begin{enumerate}
\item \lstinline{sudo apt-key adv --keyserver hkp://keyserver.ubuntu.com:80 --recv 7F0CEB10}
\item \lstinline{echo 'deb http://downloads-distro.mongodb.org/repo/ubuntu-upstart dist 10gen' | sudo tee /etc/apt/sources.list.d/mongodb.list}
\item \lstinline{sudo apt-get update}
\item \lstinline{sudo apt-get install mongodb-10gen}
\item \lstinline{sudo service mongodb start}
\end{enumerate}

\item A NodeJs környezet telepítése: \lstinline{sudo apt-get install nodejs}
\item Az alkalmazás telepítése:
\begin{enumerate}
\item \lstinline{git clone git@github.com:chiller/mint.git}
\item \lstinline{cd mint/}
\item \lstinline{npm install}
\item 
\end{enumerate}
\end{enumerate}

\clearpage\section{Az alkalmazás bemutatása}

public url

%----------------------------------------------------------------------------
\clearpage\section{Az állapotgép kód generálása}
%----------------------------------------------------------------------------

\begin{figure}[!ht]
\centering
\includegraphics[width=100mm, keepaspectratio]{figures/tanulo.png}
\caption{A bemeneti diagram} 
\end{figure}


\begin{lstlisting}[caption=Az állapotgép kódgeneráló UnderscoreJS template-je]  

var sm = function() {

states = [

<% doc.entities.forEach(function(entity,i){
%>
  {"name": "<%= entity.title %>", 
   <%if(i==0){ %> "initial":true, <% }%>
  "events": {
    <% entity.connections_from.forEach(function(con){%>
        "<%=con.label%>":"<%=con.to%>",
    <%}) %>}
},<%
}) %>

]

 function StateMachine(states){
        this.states = states;
        this.indexes = {}; 
        for( var i = 0; i< this.states.length; i++){
            this.indexes[this.states[i].name] = i;
            if (this.states[i].initial){
                this.currentState = this.states[i];
            }
        }
        this.consumeEvent = function(e){
            if(this.currentState.events[e]){
                this.currentState = this.states[this.indexes[this.currentState.events[e]]] ;
                console.log(this.currentState.name);
            }
        }
        this.getStatus = function(){
            return this.currentState.name;
        }
    }
    return new StateMachine(states);
}

\clearpage

\end{lstlisting}



\begin{lstlisting}[caption=A generálás eredménye] 

var sm = function() {

states = [


  {"name": "Jatszik", 
    "initial":true, 
  "events": {"almos":"Alszik","holnapvizsga":"Tanul",}
},
  {"name": "Tanul", 
   
  "events": {"unatkozik":"Jatszik",}
},
  {"name": "Alszik", 
   
  "events": {"felebred":"Kave",}
},
  {"name": "Kave", 
   
  "events": {"holnapvizsga":"Tanul","almos":"Kave",}
},

]

 function StateMachine(states){
        this.states = states;
        this.indexes = {}; 
        for( var i = 0; i< this.states.length; i++){
            this.indexes[this.states[i].name] = i;
            if (this.states[i].initial){
                this.currentState = this.states[i];
            }
        }
        this.consumeEvent = function(e){
            if(this.currentState.events[e]){
                this.currentState = this.states[this.indexes[this.currentState.events[e]]] ;
                console.log(this.currentState.name);
            }
        }
        this.getStatus = function(){
            return this.currentState.name;
        }
    }
    return new StateMachine(states);
}

\end{lstlisting}
