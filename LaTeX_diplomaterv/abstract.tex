%----------------------------------------------------------------------------
% Abstract in hungarian
%----------------------------------------------------------------------------
\chapter*{Kivonat}\addcontentsline{toc}{chapter}{Kivonat}

Napjainkban a böngészőben futó webalkalmazások már sok teret hódítottak helyettesítve az asztali alkalmazásokat, hiszen könnyebb a telepítésük, nem kell tárhellyel spórolni és a felhőalapú megoldások bizonyos értelemben biztonságban helyezik az adatainkat. A webes diagramszerkesztő egy jó példa erre, mivel pár percen belül már akár kollaboratívan dolgozhatunk egy diagramon távoli munkatársunkkal.

A szakdolgozatom célja egy webes modellszerkesztő alkalmazás, ami alkalmas vizuális programozásra és kódgenerálásra és ezen felül kollaboratívan lehessen dolgozni benne. Ez a cél mellett a kiválasztott technológiákat is meg kellet ismernem és az elkészült alkalmazást teljesítmény szempontjából értékeltem.

A Javascript reneszánszát éli napjainkban. Már rég nem csak arra való, hogy a böngészőben dinamikus layout-ot létrehozzunk, eljött az a pont, ahol kliens-, szerver- és adatbázis szinten -- idegen szóval -- \emph{full-stack} képességei vannak. A szakdolgozatom keretein belül meg akartam vizsgálni, hogy mennyire érett a technológia komplex webalkalmazások fejlesztésére. 

Munkám eredménye egy kollaboratív diagramszerkesztő, amiben egy kódtranszformációs template-et is megadva tetszőleges nyelvű kódot lehet generálni, Javascript esetében ezt a kódot fel is lehet használni más rendszerben. A kiválasztott technológia alkalmas volt hatékony prototipizálásra, modularitásra és elegáns megoldások implementálására.  

\vfill

%----------------------------------------------------------------------------
% Abstract in english
%----------------------------------------------------------------------------
\chapter*{Abstract}\addcontentsline{toc}{chapter}{Abstract}

Ha jó a magyar, megírom az angolt.
\vfill

